
\setchapterpreamble[u]{\margintoc}
\chapter{Linear regression}
\label{chap:linear_regression}

We observe iid pairs $(X_1, Y_1), \ldots, (X_n, Y_n)$ where $X_i \in \R^d$ and $Y_i \in \R$.
We want to \emph{predict} $Y_i$ from $X_i$.
Actually, we want to study the conditional distribution $Y_1 | X_1$, namely we want to \emph{regress} $Y_i$ on $X_i$.
We know that the closest $X_1$-measurable function from $Y_1$ is the conditional expectation $\E (Y | X) = f(X)$  in $L^2$, but we do not know the joint distrivbution of $(X_1, Y_1)$
So we want to use the observations $(X_i, Y_i)$ in order to build some approximation of $f$.

On the other hand, wat kind of functions could we consider ?
As a start, we should consider the simplest non-constant function $\R^d \go \R$ we can think of, which is naturally a \emph{linear} function $f(x) = x^\top \theta + c$.

\begin{kaobox}[frametitle=Features engineering]
	Considering only linear function is of course very limiting. But let us stress that, in practice, we can do whatever we want with the data $(X_i, Y_i)$. A linear model is typically \emph{trained} on \emph{mappings} of $X_i$, that can include non-linear mappings such as a polynomial mapping, pairwise products leading to $d(d-1)/2$ extra coordinates, etc. The construction of such a \emph{features mapping} is called \emph{feature engineering} in statistical and machine learning, and is more an art than a science. Note that many industrial large scale problems (such as web-display advertisment) are handled using such linear methods, but on highly tested and engineered features. We won't discuss this in this chapter, and will assume that $X_i$ are well crafted features vectors on which we want to train a linear method.	
	This makes the model linear we respect to $\phi(X_i)$ but certainly not with respect to $X_i$.
	We forget about it and work on $X_i$ assuming that...
\end{kaobox}

Also, to simplify notations we will forget about the \emph{intercept}, also called \emph{population bias} $b \in \R$ since we can, without loss of generality simply replace $\theta$ by $[1 \; \theta^\top]^\top$ and $X_i$ by $[1 \; X_i^\top]^\top$ (by $d$ by $d=1$).

We assume that $\E(Y_i | X_i)$ can reasonably well approximated by $\theta^\top X_i$ for some $\theta \in \R^d$ to be trained and 

We therefore assume that we can write $Y_i = \theta^\top X_i + \eps_i$ where $\eps_i = Y_i - \E(Y_i | X_i)$ and $\E(\eps_i | X_i) = 0$.

\todo{dire que c'est une definition ce modele lineaire et qu'est ce que l'hypothse}


\section{Least squares estimator} % (fold)
\label{sec:least_squares_estimator}

We say that iid real random variables $\eps_1, \ldots, \eps_n$ is the \emph{noise} of the linear model $Y_i = X_i^\top \theta + \eps_i$.
How can we \emph{estimate} or \emph{train} $\theta \in \R^d$ ?
A natural idea is to find $\wh \theta_n \in \R^d$ such that $X_i^\top \wh \theta_n$ is close to $Y_i$ for each $i=1, \ldots, n$. The simplest way to measure this closeness is to use the Euclidean distance on $\R^n$.
Let us first introduce the following notations:
\begin{equation*}
	\by = \begin{bmatrix}
		Y_1 \\
		\vdots \\
		Y_n
	\end{bmatrix}
	\quad \text{and} \quad
	\bX = 
	\begin{bmatrix}
		X_{1, 1} & \cdots & X_{1, d} \\
		\vdots & \ddots & \vdots \\
		X_{n, 1} & \cdots & X_{n, d}
	\end{bmatrix}	
	=
	\begin{bmatrix}
		X_1^\top \\
		\vdots \\
		X_n^\top 
	\end{bmatrix}
	= 
	\begin{bmatrix}
		X^1 & \cdots & X^d
	\end{bmatrix}	
	\in \R^{n \times d}
\end{equation*}
and we will denote also $\eps = [\eps_1 \cdots \eps_n]^\top \in \R^n$ the noise vector 

There notation $X^j$ for the $j$-th columns of $\bX$ and $X_i$ is the $i$-th row of $\bX$.

\sidenote{All vectors are written as column matrices BLABLA The norm $\norm{\cdot}$ stands for the Euclidean norm}

A \emph{least squares} estimator or \emph{ordinary least squares} is defined as
\begin{equation}
	\label{eq:least-squares-estimator}
	\wh \theta_n \in \argmin_{\theta \in \R^d} \norm{\by - \bX \theta}^2 = \argmin_{\theta \in \R^d} \sum_{i=1}^n (Y_i - X_i^\top \theta)^2
\end{equation}
How can we compute $\wh \theta_n$ ?
The definition of $\wh \theta_n$ given by Equation~\eqref{eq:least-squares-estimator} entails that
\begin{equation*}
	\bX \wh \theta_n = \proj_V (Y)
\end{equation*}
where $\proj_V$ is the orthogonal projection operator onto $V = \{ \bX u : u \in \R^d \} = \spa(\bX) = \spa(X^1, \ldots, X^d)$, the linear space in $\R^n$ spanned by the columns of $\bX$.
\todo{insert picture}
We need to have that $Y - \bX \wh \theta_n \perp V$, namely $\inr{\bX u, Y - \bX \wh \theta_n} = 0$ for any $u \in \R^d$, namely $u^\top \bX^\top (Y - \bX \wh \theta_n) = 0$ for any $u$ which leads to the so-called \emph{normal equation}
\begin{equation}
	\label{eq:least-squares-normal-equation}
	\bX^\top \bX \wh \theta_n = \bX^\top \by.
\end{equation}
This means that $\wh \theta_n$ is a solution to this normal equation, which in turn is simply a linear system.
Another approach that leads to the same result is to put $F(\theta) = \norm{Y - \bX \theta}^2$, which is a \emph{convex} and differentiable function on $\R^d$, so that $\wh \theta_n$ is an $\argmin$ if and only if it is a critical point, namely $\grad F(\wh \theta_n) = 0$, but $\grad F(\theta) = 2 \bX^\top (\bX \theta - \theta)$ so that we end up with the same normal Equation~\eqref{eq:least-squares-normal-equation}.

A least squares estimator is therefore solution tothe linear system???

From now on, let us assume that $\E \norm{X}^2 < +\infty$ and $\E(Y^2) < +\infty$.
This allows to define the $d \times d$ matrix
\begin{equation}
	\E(X X^\top) = (\E(X_j X_k))_{1 \leq j, k \leq d}
\end{equation}
and let us recall that the covariance matrix between two random vectors $U$ and $V$ is, whenever it makes sense, given by
\begin{equation*}
	\cov(U, V) = \E [(U - \E U)(V - \E V)^\top]
\end{equation*}
\todo{les matrices avec du gras !!!}
and we will denote $\var (U) = \cov(U, U)$ the covariance matrix of $U$.
Let us also remark that whenever $V = A U + b$ for some deterministic matrix $A$ and vector $b$ we have that $\E [V] = A \E [U] + b$ and $\var[V] = A \var(U) A^\top$.
Also, we will write $\bA \succ 0$ to say that a matrix $\bA$ is positive definite.
\begin{theorem}
	\label{thm:least-squares-existence}
	Let $n \geq d$. The following points about $\P_X$ are all equivalent whenever $X_1, \ldots, X_n$ are independent.
	\begin{enumerate}
		\item $\E(X X^\top) \succ 0$
		\item For any hyperplane $H \subset \R^d$ we have $\P(X \in H) = 0$, namely $\P(X^\top \theta = 0) = 0$ for any $\theta \in S^{d-1}$
		\item $\bX^\top \bX = \sum_{i=1}^n X_i X_i^\top \succ 0$ almost surely
		\item The least squares estimator is uniquely defined and given by
		\begin{equation*}
			\wh \theta_n = (\bX \bX)^{-1} \bX^\top \by
		\end{equation*}
		almost surely.
	\end{enumerate}
	Whenever $\P_X$ satisfies either of this points, we say that $\P_X$ is \emph{non-degenerate}.
\end{theorem}
Theorem~\ref{thm:least-squares-existence} explains exaclty when the least squares estimatpr is 

\begin{proof}
	Point~(3) $\Leftrightarrow$ Point~(4) is obvious since $\bX^\top \bX \succ 0$ entails that $\bX^\top \bX$ is invertible. Point~(1) $\leftrightarrow$ Point~(2) is obvious as well, since $0 = \E(X X^\top) u = 0$ entails $0 = u^\top \E(X X^\top) u = \E[ (X^\top u)^2] = 0$ which entails $X^\top u = 0$ almost surely. Point~(3) $\Rightarrow$ Point~(2) comes from a proof by contradiction. If $0 < p = \P(X^\top u = 0)$ then $X_i^\top u = 0$ for all $i=1, \ldots, n$ with a probability $p^n > 0$ since $X_1, \ldots, X_n$ are iid, so that $\bX^\top \bX \theta = \sum_{i=1}^n (X_i^\top \theta)^2 X_i = 0$ and $\bX^\top \bX$ cannot be invertible almost surely.
	The proof of Point~(2) $\Rightarrow$ Point~(3) can be done by recurrence. We first remark that $\bX^\top \bX$ is invertible if and only if $\spa(X_1, \ldots X_n) = \R^d$ (indeed $\ker(\bX^\top \bX = \ker(\bX)$ so that $\bX^\top \bX u = 0 \Leftrightarrow \bX u = 0 \Leftrightarrow X_i^\top \theta = 0$ for all $i=1, \ldots, n$.) We will show that $\spa(X_1, \ldots, X_d) \R^d$ almost surely by recurrence. We put $V_k = \spa(X_1, \ldots, X_k)$ so that $\dim(V_k) \leq k \leq d$. For $k=1$ we do have $\dim V_1 = 1$ so it is OK. Assume that $\dim(V_{k-1}) = k-1$. We have that $X_k$ is indendent from $V_{k-1} = \spa(X_1, \ldots, X_{k-1})$ and $\dim(V_{k-1}) = k-1 < d$ so that $V_{k-1} \subset H$ where $H \subset \R^d$ is an hyperplane. So, we have again by indepedence that $\P(X_k \in V_{k-1}) = \P(X_k \in V_{k-1} | X_1, \ldots, X_{k-1}) \leq \P(X_k \in H) = 0$ using Point~(2). So, $X_k \notin V_{k-1}$ almost surely, and $\dim(V_k) = k$ almost surely.
\end{proof}

\section{Some properties of the least squares estimator} % (fold)
\label{sec:some_properties_of_the_least_squares_estimator}

In this section, we will simplify further the problem, and assume that everything will be computed conditionally on $X_1, \ldots, X_n$, so that we will threat them as if these were deterministic random vectors.
We assume also that $\eps_1, \ldots, \eps_n$ are iid and such that $\E[\eps_i] = 0$ and $\var[\eps_i] = \sigma^2 < +\infty$, which means that $\E[\beps] = 0$ and $\var[\beps] = \sigma^2 \bI_n$, so that
\begin{equation*}
	Y_i = X_i^\top \theta + \eps_i.
\end{equation*}
We assume also that $\bX^\top \bX \succ 0$ (namely $\bX$ is full rank, we saw in Theorem~\ref{thm:least-squares-existence} what this means whenever $X_i$ are random).
In this case we can write $\wh \theta_n = (\bX^\top \bX)^{-1} \bX^\top \by$ so that
\begin{equation*}
	\E_\theta[\wh \theta_n] = (\bX^\top \bX)^{-1} \bX^\top \E_\theta[\by] = (\bX^\top \bX)^{-1} \bX^\top \bX \theta = \theta
\end{equation*}
which means that $\wh \theta_n$ is an \emph{unbiased} estimator. We can write also
\begin{equation*}
	\var_\theta[\wh \theta_n] = \var_\theta[ (\bX^\top \bX)^{-1} \bX^\top \by] = \bA \var_\theta(\by) \bA^\top = \sigma^2 \bA \bA^\top = \sigma^2 (\bX^\top \bX)^{-1}
\end{equation*}
where we put $\bA = (\bX^\top \bX)^{-1} \bX^\top$.
In particular, this proves that the quadratic risk of $\wh \theta_n$ is given by
\begin{equation*}
	\E_\theta[(\wh \theta_j - \theta_j)^2] = \var_\theta[\wh \theta_j] = \sigma^2 ((\bX^\top \bX)^{-1})_{j, j}
\end{equation*}
or equivalently that
\begin{equation*}
	\E_\theta \norm{\wh \theta - \theta}^2 = \sigma^2 \tr [(\bX^\top \bX)^{-1}].
\end{equation*}
We don't know either the variance $\sigma^2$ of the noise, so we want to build an estimator for it as well.
It is pretty natural to look at the random vector $\wh \beps := \by - \bX \wh \theta$ called the \emph{residual vector}. 
Note also that $\bX \wh \theta = \proj_V(\by) = \bH \by$ for the matrix $\bH := \bX (\bX^\top \bX)^{-1} \bX^\top$ that we call the \emph{hat matrix} since $\wh \by := \bX \wh \theta = \bH \by$ (it puts a hat on $\by$).
The matrix $\bH$ is the projection matrix onto the set $V = \spa(\bX^1, \ldots, \bX^d) = \im(\bX)$.
This leads to
\begin{equation*}
	\by - \bX \wh \theta = (\bI - \bH) \by = (\bI - \bH) (\bX \theta + \eps) = (\bI - \bH) \beps
\end{equation*}
since $\bI - \bH$ is the projection matrix onto $V^\perp$ (the space orthogonal to $V$) and since obviously $\bX \theta \in V$.
Finally, we get
\begin{equation*}
	\E_\theta \norm{\wh \theta - \theta}^2 = \E_\theta \norm{ (\bI - \bH) \beps}^2 
	= \tr \E [ (\bI - \bH) \beps \beps^\top (\bI - \bH)] = \tr ( (\bI - \bH) \E [ \beps \beps^\top ] (\bI - \bH) = \sigma^2 \tr( (\bI - \bH)^2 ) = \sigma^2 \tr( \bI - \bH ) = \sigma^2 (n - d),
\end{equation*}
where we used that $\bI - \bH$ is the projection matrix onto the space $V^\perp$ of dimension $n-d$, so that $(\bI - \bH)^2 = \bI - \bH$ and $\tr(\bI - \bH) = n-d$.
What we just proved is that
\begin{equation*}
	\wh \sigma^2 := \frac{1}{n - d} \norm{\by - \bX \wh \theta}^2 = \frac{1}{n - d} \sum_{i=1}^n (Y_i - X_i^\top \wh \theta)^2
\end{equation*}
is an \emph{unbiased} estimator of $\sigma^2$.
Now, if we want to go further about these estimators, we need some extra structure, in particular if we want to study the distribution of $\wh \theta$ and $\wh \sigma^2$. 
To do so, we assume that the noise vector $\beps$ is Gaussian, as performed in the next section.

\section{Gaussian linear model} % (fold)
\label{sec:gaussian_linear_model}


We keep the same setting as in the previous Section~\ref{sec:some_properties_of_the_least_squares_estimator} but assume furthermore that $\eps_1, \ldots, \eps_n$ are iid and such that $\eps_i \sim \nor(0, \sigma^2)$.
This means that $\beps$ is a Gaussian vector with multivariate Gaussian distribution $\beps \sim \nor(0, \sigma^2 \bI)$.
Let us first start this section with some reminders about Gaussian vectors.

\paragraph{Gaussian vectors.} % (fold)

We say that a random vector $Y \in \R^n$ is \emph{Gaussian} whenever $\inr{u, Y}$ is a Gaussian real random variable for any $u \in \R^d$.
Moreover, if $\bSigma = \var(Y) \succ 0$ and $Y$ is a Gaussian vector, then $Y \sim \nor(\mu, \bSigma)$ where $\mu = \E[Y]$ and has density
\todo{transformation linear est lineaire, meme si la matrice est degenereea lors on ecrti quand meme ..}
\begin{equation*}
	f_Y(y) = \frac{1}{\sqrt{(2 \pi)^d \det \bSigma}} \exp \Big( - (x - \mu)^\top \bSigma^{-1} (x - \mu) \Big)
\end{equation*}
with respect to the Lebesgue measure on $\R^n$.
Note that if $Z \sim \nor(0, \bI_n)$, in this case with say that $Z$ is \emph{standard Gaussian}) then $Y$ has the same distribution as $\bSigma^{1/2} Z + \mu$.
Also, if $Y \sim N(\mu, \bSigma)$ where $\bSigma$ is a diagonal matrix with positive entries, then the coordinates of $Y$ are independent.
Note also that if $Z \sim \nor(0, \bI)$ and $\bQ$ is an orthonormal matrix ($\bQ^\top \bQ = \bI$) then $\bQ Z \sim \nor(0, \bI)$\todo{isotropic blabla}

\paragraph{About the $\gam$ distribution.} % (fold)

% paragraph about_the_ (end)
Gamma distribution. We say that a real random variable $G$ has $\gam(a, \lambda)$ distribution, where $a > 0$ is called the \emph{shape} parameter and $\lambda > 0$ is called the \emph{intensity} parameter whenever $G$ has density
\begin{equation*}
	f_{a, \lambda}(x) = \frac{\lambda^a}{\Gamma(a)} x^{a - 1} e^{-\lambda x} \ind{x \geq 0}
\end{equation*}
with respect to the Lebesgue measure on $\R$.\todo{recall Gamma function}
Note that if $G \sim \gam(a, \lambda)$ then $\E[G] = a / \lambda$ and $\var[G] = a / \lambda^2$ and note that $\mode(G) = (a - 1) / \lambda$ if $a > 1$.\todo{what is the mode ?}
Whenever $G_1 \sim \gam(a_1, \lambda)$ and $G_2 \sim \gam(a_2, \lambda)$ are independent random variable, then $G_1 + G_2 \sim \gam(a_1 + a_2, \lambda)$. \todo{note about convolution, stable distribution by convolutions, etc...}

Also, if $E_1, \ldots, E_n$ are iid distributed as $\exp(\lambda)$ then $\sum_{i=1}^n E_i \sim \gam(n, \lambda)$.

\paragraph{About the $\chisq(n)$ distribution.} % (fold)

If $p \in \N \setminus \{ 0 \}$ then $\chisq(p) = \gam(p/2, 1/2)$ is called the \emph{Chi squared distribution with $p$ degrees of freedom}.
Although being a particular case of $\gam$ distribution, the $\chisq$ distribution is important in statistics, since it is the distribution of the squared Euclidean norm of a standard Gaussian vector.
Indeed, if $Z \sim \nor(0, \bI_n)$ then $\norm{Z}^2 = \sum_{i=1}^n Z_i^2 \sim \chisq(n)$.
This easiliy comes from the fact that $Z_i^2 \sim \gam(1/2, 1/2)$ so that by independence $\norm{Z}^2 \sim \gam(n/2, 1/2) = \chisq(n)$.
The density of $\chisq(n)$ is therefore 
\begin{equation*}
	f_n(x) = \frac{2^{-n / 2}}{\Gamma(n/2)} x^{n / 2 - 1} e^{-x / 2} \ind{x \geq 0}.
\end{equation*}
with respect to $\leb(\R^n)$.

\paragraph{About the student $t(n)$ distribution.} % (fold)


If $U \sim \nor(0, 1)$ and $V \sim \chisq(p)$ are independent random variables, then
\begin{equation*}
	\frac{U}{\sqrt{V / n}} \sim t(n)
\end{equation*}
where $t(p)$ is called the \emph{student distribution when $p$ degrees of freedom}.\todo{guiness story}, that has density
\begin{equation*}
	f_n(x) = \frac{1}{\sqrt{n \pi}} \frac{\Gamma((n+1) / 2)}{\Gamma(n/2)} \frac{1}{(1 + x^2 / n)^{(n + 1)/2}}
\end{equation*}
with respect to the Lebesgue density. We have that $\E[T] = 0$ and $\var(T) = n / (n - 2)$ whenever $n > 2$ if $T \sim \stu(n)$.

It can be seen that $\stu(n) \gosto \nor(0, 1)$ whenever $n \rightarrow +\infty$ since ???

\paragraph{About the $\fis(p, q)$ distribution} % (fold)

Let $p, q \in \N \setminus \{ 0 \} $. If $U \sim \chisq(p)$ and $V \sim \chisq(q)$ are independent then
\begin{equation*}
	\frac{U / p}{V / q} \sim \fis(p, q)
\end{equation*}
where $\fis(p, q)$ stands for the \emph{Fisher distribution} which has density
\begin{equation*}
	f_{p, q}(x) = \frac{1}{x \beta(p/2, q/2)} \Big( \frac{px}{px + q} \Big)^{p/2} \Big(1 - \frac{px}{px + q} \Big)^{q/2}
\end{equation*}
where $\beta(a, b) = \Gamma(a) \Gamma(b) / \Gamma(a + b) = \int_0^1 t^{a-1} (1 - t)^{b - 1} dt$.

\paragraph{About the $bet(a, b)$ distribution.} % (fold)

If $G_1 \sim \gam(a, \lambda)$ and $G_2 \sim \gam(b, \lambda)$ are independent then
\begin{equation*}
	\frac{G_1}{G_1 + G_2} \sim \bet(a, b)
\end{equation*}
where $\bet(a, b)$ is called the Beta distribution, which has density
\begin{equation*}
	f_{a, b}(x) = \frac{1}{\beta(a, b)} x^{a - 1} (1 - x)^{b - 1} \ind{[0, 1]}(x)
\end{equation*}
with respect to the Lebesgue measure on $\R$. Note that if $B \sim \beta(a, b)$ then $\E[B] = \frac{a}{a + b}$, that $\var[B] = ab / ((a + b)^2 (a + b + 1))$ and $\mode(B) = (a - 1) / (a + b - 2)$ whenever $a, b > 1$ \todo{check these formulas}.


\paragraph{Back to the Gaussian vectors.}

Let us now provide some stuff about Gaussian vectors, that will prove important for the Gaussian linear model. Indeed, thanks to the additional strucutre we put blablabla

\begin{theorem}
	Let $Z \sim \nor(0, \bI)_n$ and let $V_1, \ldots, V_k$ be orthogonal linear spaces of $\R^n$. Define the Gaussian vectors $Z_j = \bP_j Z := \proj_{V_j}(Z)$ namely $\bP_j$ is the orthonormal projection matrix onto the space $V_j$. Then, we have that $Z_1, \ldots, Z_k$ are independent random vectors, and that
	\begin{equation}
		\norm{Z_j}^2 \sim \chisq(n_j)
	\end{equation}
	where $n_j = \dim(V_j)$ (note that $\sum_{j=1}^k n_j \leq n$).
\end{theorem}

\begin{proof}
	We have $\var[Z_j] = \bP_j \bP_j^\top = \bP_j$ since $\bP_j$ is an orthogonal projection matrix and $Z_j = \bP_j Z$, which entails that $Z_j$ is a Gaussianv ector (as lienar transformation of a gaussian vector) and that $Z_j \sim \nor(0, \bP_j)$. Note that $Z_j$ has no density with respect to the Lebesgue measure, it is a random vector on $\R^n$ which belongs to linear space of dimension $n_j < n$ BLABLA. Now, we habe
	\begin{equation*}
		\cov(Z_j, Z_{j'}) = \cov(\bP_j Z, \bP_{j'} Z) = \bP_j \bP_{j'}^\top = \bO
	\end{equation*}
	since $V_j \perp V_{j'}$, so that $Z_j$ and $Z_{j'}$ are independent random vectors, since the covariance matrix is block diagonal \todo{faudrait l'epxliquer qq part}.
	This proves that the $Z_1, \ldots, Z_k$ are independent random vectors.
	Finally, since $\bP_j$ is an orthogonal projection matrix onto a space of dimension $n_j$, we can decompose \todo{why ?} it as $\bP_j = \bQ \bD_{n_j} \bQ^\top$ where $\bD_{n_j} = \diag[1, \ldots, 1, 0, \ldots, 0]$ is the diagonal matrix with first $n_j$ diagonal elements equal to $1$ and all others equal to $0$.
	We know from ??? that $Z' := \bQ^\top Z \sim \nor(0, \bI_n)$ so that $\bP_j Z = \bQ [Z_1' \cdots Z_{n_j}']^\top =: \bQ Z_-'$ and $\norm{\bP_j Z}^2 = \norm{\bQ Z_-''}^2 = \norm{Z_-''}^2$ (since $\bQ^\top \bQ = \bI_n$) so that $\norm{\bP_j Z}^2 = \sum_{j=1}^{n_j} (Z_j')^2$ so that $\norm{\bP_j Z}^2 \sim \chisq(n_j)$ since $Z' \sim \nor(0, \bI_n)$. This concludes the proof of the theorem.
\end{proof}

\paragraph{Back to the gaussian linear model.}

Recal that awe have $\by = \bX \theta + \beps$ with $\beps \sim \nor(0, \sigma^2 \bI_n)$.
We have now the right tools to give the full distribution of both estimators $\wh \theta$ and $\wh \sigma^2$.
\begin{theorem}
	Assume that $\rank(\bX) = d$ that $\by = \bX \theta + \beps$ and that $\beps \sim \nor(0, \sigma^2 \bI_n)$. Put $\wh \by = \bX \wh \theta$ where $\wh \theta = (\bX^\top \bX)^{-1} \bX^\top \by$  and $\wh \sigma^2 = \norm{\by - \wh \by}^2 / (n-d)$. Then we have that $\wh \theta$ and $\wh \sigma^2$ are \emph{independent}, that and that
	\begin{equation*}
		\wh \theta \sim \nor(\theta, \sigma^2 (\bX^\top \bX)^{-1}), \quad \norm{\by - \wh \by}^2 / \sigma^2 \sim \chisq(n - d) \quad \text{and} \quad \norm{\bX( \wh \theta - \theta)}^2 / \sigma^2 \sim \chisq(d).
	\end{equation*}
\end{theorem}

\begin{proof}
	We know from before that $\by - \wh \by = \proj_{V^\perp}(\by) = \proj_{V^\perp}(\beps)$
	 and that $\bX (\wh \theta - \theta) = \proj_V(\by) - \bX \theta = \proj_V(\beps)$
	 so that, since $V \perp V^\perp$ and $\dim V = d$ and $\dim V^\perp = n - d$, the conclusion follows from the Cochran theorem ???.
\end{proof}

Theorem~??  has many consequences. For instance if $\sigma^2$ is known, then the set
\begin{equation*}
	\cE = \Big \{ \theta \in \R^d : \frac{1}{\sigma} \norm{(\bX^\top \bX)^{-1} (\wh \theta - \theta)}^2 \leq q_{\chisq(d)}(1 - \alpha)  \Big\}
\end{equation*}
where $q_{\chisq(d)}(1 - \alpha)$ is the quantile function of the $\chisq(d)$ distribution at $1 - \alpha$, if a confidence ellipsoid for $\theta$ in the Gaussian linear model at level $1 - \alpha$, since it satisfies by construction the coverage property $\P_\theta[ \theta \in \cE] = 1 - \alpha$.
Whenever $\sigma^2$ is unknown (which is always the case) then using the fact that
\begin{equation*}
	\frac{\norm{\bX(\wh \theta - \theta)}^2 / d}{\norm{\by - \bX\wh \theta}^2 / (n - d)} \sim \fis(d, n-d)
\end{equation*}
entails that the ellipsoid
\begin{equation*}
	\Big \{ \theta \in \R^d : \frac{1}{\wh \sigma^2} \norm{(\bX^\top \bX)^{-1} (\wh \theta - \theta)}^2 \leq q_{\fis(d, n - d)}(1 - \alpha)  \Big\}
\end{equation*}
if a confidence region at level $1 - \alpha$.
Both confidence regions provide coverage for the whole vector $\theta$.
We can also build confidence intervals for each coordinate of $\theta$.
Indeed, we have $\theta_j = \theta^\top e_j$ where $e_j$ is the canonical basis vector with $1$ at cordinate $j$ and $0$ elsewhere. More generally we can build a confidence interval for $a^\top \theta$ for any vector $a \in \R^d$ by looking for an ancillary statistic. We know that $a^\top(\wh \theta - \theta) \sim \nor(0, \sigma^2 a^\top (\bX^\top \bX)^{-1} a)$ and that $\wh \theta$ and $\wh \sigma^2$ are independent, so that
\begin{equation*}
	\frac{a^\top(\wh \theta - \theta)}{\sigma \sqrt{a^\top (\bX^\top \bX)^{-1} a}} \sim \nor(0, 1) \text{ and } \frac{(n - d) \wh \sigma^2}{\sigma^2} \sim \chisq(n^d).
\end{equation*}
This entails that 
\begin{equation*}
	\frac{a^\top(\wh \theta - \theta)}{\sqrt{\wh \sigma^2 a^\top (\bX^\top \bX)^{-1} a}} \sim \stu(n - d)
\end{equation*}
as the ratio between independent $\nor(0, 1)$ and $\chisq(n - d)$ random variables.
This leads to the coverage property
\begin{equation*}
	\P_{\theta, \sigma^2} \bigg\{ a^\top \theta \in  \Big[  a^\top \wh \theta \pm \sqrt{\wh \sigma^2 a^\top (\bX^\top \bX)^{-1} a} q_{\stu(n-d)}(1 - \alpha/2) \Big] \bigg\} = 1 - \alpha
\end{equation*}
in particular for $a = e_j$ we obtain that
\begin{equation*}
	\P_{\theta, \sigma^2} \bigg\{ \theta_j \in  \Big[  a^\top \wh \theta \pm \sqrt{\wh \sigma^2 (\bX^\top \bX)_{j, j}^{-1}} q_{\stu(n-d)}(1 - \alpha/2) \Big] \bigg\} = 1 - \alpha
\end{equation*}
This allows in particular to build a test for the hypothesis $H_{0, j} : \theta_j = 0$ versus $H_{1, j} : \theta_j \neq 0$.
A confidence interval for $\sigma^2$ can be also easily built using the ancillary statistic $(n - d) \wh \sigma^2 / \sigma^2 \sim \chisq(n - d)$.

\paragraph{Fisher test.} % (fold)

Using the previous confidence interval we can test $H_0 = \theta_1 = \theta_2$ by using $a = [1, -1, 0, \ldots, 0]$. But, how can we test $H_0 : \theta_1 = \theta_2 = 0$ or more generally a \emph{multiple} null hypothesis such as
\begin{equation*}
	H_0 : \theta_1 = \cdots = \theta_k = 0
\end{equation*}
for fomr $k = 2, \ldots, d$ ?
An approach can rely on the use of separate tests with siple ?? null hypotheses $H_{0, j} : \theta_j = 0$ for $j=1, \ldots, k$, with rejection sets $R_{j, \alpha}$ that satisfy $\P_{\theta_j = 0}[R_{j, \alpha}] \leq \alpha$ for each $j=1, \ldots, k$.
So, an approach to test $H_0$ given by ??? is to consider a rejection set given by the union of these individual rejection sets, but with decreased levels up to $\alpha / k$, since using the union bound
\begin{equation*}
	\P_{\theta_1 = \cdot \theta_k = 0} \Big [  \cup_{j=1}^k R_{j, \alpha / k} \Big] 
	\leq \sum_{j=1}^k \P_{\theta_j = 0} [ R_{j, \alpha / k} ] \leq k \times \alpha / k = \alpha,
\end{equation*}
so that the test with rejection set ??? for the hypothesis ??? has, indeed, level $\alpha$
This is called the Bonferroni correctino, which is the most simple approach for multiple testing ?? more on that later ??? \todo{histoire avec Olive Jean DUnon ???}

This approach relies on decreasing the individua levels of each $\alpha$ by $\alpha / k$, which can be a strong decrease whenever $k$ is large, which deteriorates a lot the power of each individual test, hence the power of the test ????
We can do much better than that in the Gaussian linear model, thanks to the Fisher test.
Let us continuer with the null assumption ??? and put $\Theta_0 = \{ \theta \in \R^d : \theta_1 = \cdots \theta_k = 0 \}$ and put again $V = \im(\bX) = \{ \bX u : u \in \R^d\}$ and put $W = \{ \bX u : u \in \Theta_0 \}$. We can consider in general $\theta_0 = \ker(\bA)$ for some matrix $\bA : ? \rightarrow ?$. In the considered example we have that $\bA$ is a $k \times d$ matrix given by 
$\bA = [\bI_k \bO_{k, d-k}]$ corresponding to the horizontal concatenation of the identity matrix on $\R^k$ and a $k \times (d-k)$ zero matrix.
We introduce a geometric solution to this testing problem: we decompose $\R^n$ as the following direct sums
\begin{equation*}
	\R^n = V^\perp \oplus V = V^\perp \oplus W \oplus W'
\end{equation*}
where we note that $\dim(V^\perp) = n - d$, $\dim(W) = d - k$ and $\dim(W') = k$, with $W = \{ \bX \theta : \theta \in \Theta_0 \}$ is a subset of $V$.
Consider now the projections $\proj_V(\by)$ and $\proj_W(\by)$ of $\by$ onto $V$ and its subspace $W$.
Pythagora's theorem entails that $\norm{\by - \proj_W(\by)}^2 = \norm{\by - \proj_V(\by)}^2 + \norm{\proj_V(\by) - \proj_W(\by)}^2$ since $\by - \proj_V(\by) \perp \proj_V(\by) - \proj_W(\by) \in V$, so that 
\begin{equation*}
	\norm{\proj_V(\by) - \proj_W(\by)}^2 = \norm{\by - \proj_W(\by)}^2 - \norm{\by - \proj_V(\by)}^2.
\end{equation*}
Note that $\proj_V(\by) = \bX \wh \theta$ where $\wh \theta$ is the ordinary least squares estimator while $\proj_W(\by) = \bX \tilde \theta$ where $\tilde \theta$ is the least squares estimator computed under $H_0$, namely $\tilde \theta = \argmin_{\theta \in \Theta_0} \norm{\by - \bX \theta}^2$.
This is where the trick of the test comes in the picture: the quantity $\proj_V(\by) - \proj_W(\by)$ behaves very differently whenever $H_0$ holds or not. Indeed, \emph{under $H_0$, namel when $\bX \theta \in W$, we have}
\begin{equation*}
	\proj_V(\by) - \proj_W(\by) = \proj_{W'} (\by) = \proj_{W'}(\bX \theta) + \proj_{W'}(\beps) = \proj_{W'}(\beps)
\end{equation*}
since in this case $\bX \theta \in W \perp W'$, while $\proj_{W'}(\bX \theta) \neq 0$ when $\theta \notin \Theta_0$.
So, under $H_0$, we have that $(\proj_V(\by) - \proj_W(\by)) / \sigma = \proj_{W'}(\beps / \sigma)$ while 
 $(\by - \proj_V(\by)) / \sigma = \proj_{V^\perp}(\beps / \sigma)$ and therefore since $W' \subset V$ we have $V^\perp \perp W'$, so again, under $H_0$ we have
 \begin{equation*}
 	\frac{\norm{\proj_V(\by) - \proj_W(\by)}^2 / k}{\norm{\by - \proj_V(\by)}^2 / (n - d)} =
 	\frac{\norm{\proj_{W'}(\beps / \sigma)}^2 / k}{\norm{\proj_{V^\perp}(\beps / \sigma)}^2 / (n-d)} \sim \fis(k, n-d),
 \end{equation*}
 since the Cochran Theroem~?? gives that both numerators and denominators are independent and distributed as $\chisq(k)$ for the numerator and $\chisq(n - d)$ for the denominator, and by the deifintion of the $\fis$ distribution.
 This can be rewritten, under $H_0$, as 
 \begin{equation*}
 	\frac{\big(\norm{\by - \bX \tilde \theta}^2 - \norm{\by - \bX \tilde \theta}^2\big) / k}
 	{\norm{\by - \bX \wh \theta}^2 / (n - d)} = \frac{\norm{\bX (\wh \theta - \tilde \theta)}^2}{k \wh \sigma^2} \sim \fis(k, n-d).
 \end{equation*}
 Note that all of this was done so that the ancillary statistic does not depend on the unknown $\sigma^2$, that cancels out thanks for the ratio.
We can conclude now that the Fisher test for the hypothesis ???? with rejection region
\begin{equation*}
	R = \bigg\{  \frac{\norm{\bX (\wh \theta - \tilde \theta)}^2}{k \wh \sigma^2}  \geq q_{\fis(k, n-d)}(1 - \alpha)  \bigg\}
\end{equation*}
has level $\alpha$, namely that $\sup_{\theta \in \Theta_0} \P_\theta[R] = \alpha$.
This test is pretty intuitive and can be understood as follows: if $\theta \in \Theta_0$ then both estimators $\wh \theta$ and $\tilde \theta$ should be close, and $\bX (\wh \theta - \tilde \theta)$ should be, consequently, ``small''. The small miracle in the Fisher test is that, in the Gaussian linear model, we can perfectly quantify how small.

\begin{example}
	Consider $Y_1, \ldots, Y_n$ iid $\nor(\mu, \sigma^2)$. This is a linear model by putting $\by = \mu \bone + \beps$ where $\beps \sim \nor(0, \sigma^2 \bI_n)$.
	In this case $\wh \mu = \bar Y_n$ (since $(\bone^\top \bone)^{-1} \bone^\top \by = n^{-1} \sum_{i=1}^n Y_i$ and $\wh \sigma^2 = \frac{1}{n-1} \norm{\by - \bar Y_n \bone}^2 =  \frac{1}{n-1} \sum_{i=1}^n (Y_i - \bar Y_n)^2$ and we know from Theorem~? that $\wh \mu$ and $\wh \sigma^2$ are independent and such that $\sqrt n (\wh \mu - \mu) / \sigma \sim \nor(0, 1)$ and $(n-1) \wh \sigma^2/\sigma^2 \sim \chisq(n-1)$ so that by definition of the $\stu(n-1)$ distribution we have
	\begin{equation*}
		\sqrt{\frac{n}{\wh \sigma^2}} (\wh \mu - \mu) \sim \stu(n-1)
	\end{equation*}
	so that we can build, using this ancillary statistic a confidence interval and tests for $\mu$ when $\sigma^2$ is unknown.
\end{example}

\begin{example}
	The \emph{simple regression} model if $Y_i = a x_i + b + \eps_i$ were $a, b \in \R$, $x_1, \ldots, x_n \in \R$ and $\eps_i \sim \nor(0, \sigma^2)$ are iid. This can be written as a linear model with
	\begin{equation*}
		\by = 
		\begin{bmatrix}
			1 & x_1 \\
			\vdots & \vdots \\
			1 & x_n
		\end{bmatrix}
		\begin{bmatrix}
			a \\
			b
		\end{bmatrix}
		+ \beps
	\end{equation*}
	with $d=2$, we can compute explicitly $\wh \theta$ and $\wh \sigma^2$ and obtain their distributions. 
\end{example}

\paragraph{Residuals and leverages.}

We know that residual vector $\wh \beps = \by - \wh \by = (\bI - \bH) \by$ is such that $\E[\wh \beps] = 0$ and $\var[\beps] = \sigma^2 (\bI - \bH)$, so that
\begin{equation*}
	\wh \eps_i \sim \nor(0, \sigma^2 (1 - h_{i,i})) \quad \text{where} \quad h_{i, i} = \bH_{i, i} = X_i^\top (\bX^\top \bX)^{-1} X_i.
\end{equation*}
We know that $h_{i, i} \in [0, 1]$ since ??
We call $h_{i, i}$ the \emph{leverage} of sample $i$. We say that $i$ has small leverage whenever $h_{i, i}$ is close to zero while we say that is as a large leverage when $h_{i, i}$ is close to $1$, since in this case the contribution of sample $i$ to the linear model is important, since $\wh \eps_i \approx 0$.
Also, we note that 
\begin{equation*}
	h_{i, i} = \frac{\partial \wh Y_i}{\partial Y_i}
\end{equation*}
since $\wh \bY = \bH \bY$, namely $\wh Y_i = \sum_{j=1}^d \bH_{i, j} Y_j$.
So, the leverage $h_{i, i}$ can be understood as a quantity that measures the ``self-sensitivity'' to its prediction, namely the influence of $Y_i$ on the computation of $\wh Y_i$.
We will see also in the next Chapter that the leverage score is a very important concept as it is deeply connected to the \emph{theoretical performance} of the least-squares estimation procedure.

\section{Optimality of the least-squares} % (fold)
\label{sec:optimality_of_the_least_squares}

While the contents of the previous section is quite classical and well-understood theory the contents of this Section is not.
It comes from very result (yet simple enough to be exposed here) from a previous PhD Student of mine ????

Let us come back to the more general case where $(X_1, Y_1), \ldots, (X_n, Y_n)$ are iid with same distribution as $(X, Y)$, with $X \in \R^d$, $Y \in \R$ and $\E \norm{X}^2 < +\infty$ and $\E(Y^2) < +\infty$and a non-degenerate distribution $\P_X$, as explained in Theorem~\ref{thm:least-squares-existence}.
We consider again the linear model (not Gaussian, the results stated here are way more general than that).
Given $\sigma^2$ and $\P_X$, we consider the following classes of distribution on $(X, Y)$:
\begin{equation*}
	\cC(\P_X, \sigma^2) = \{ P_{X, Y} : X \sim P_X, Y = X^\top \theta^* + \eps \text{ for some } \theta^* \in \R^d \text{ and } \E(\eps | X) = 0, \E(\eps^2 | X) \leq \sigma^2 \text{ almost surely }\}
\end{equation*}
and the class
\begin{equation*}
	\cG(P_X, \sigma^2) = \{ P_{X, Y} : X \sim P_X, Y = X^\top \theta^*, \eps | X \sim \nor(0, \sigma^2) \}.
\end{equation*}
The first class is a general class of joint distribution on $(X, Y)$ with fixed marginal distribution $P_X$ and such that $Y$ is a linear function of $X$ plus a noise $\eps$ which is conditionally centered and with finite variance. The set $\cG(P_X, \sigma^2) \subset \cC(\P_X, \sigma^2)$ is the same as $\cC(\P_X, \sigma^2)$, but where we assume that noise to be centered Gaussian.
\todo{sidenote sur le fait que pour caracteriser la loi jointe on peut caracteriser une marginale et la loi conditionnelle}

We consider the quadratic risk \todo{differs from the one in ??? here it is  predition blabla} given by
\begin{equation*}
	R(\theta) = \E [ (Y - X^\top \theta)^2].
\end{equation*}
If $P_X$ is non degenerate and is $\Sigma = \E[ X X^\top] \succ 0$ then it is easy to see that
\begin{equation*}
	\theta^* = \argmin_{\theta \in \R^d} R(\theta) = \Sigma^{-1} \E(Y X).
\end{equation*}
Our aim here is to find a function $\wh \theta$ of $(X_1, Y_1), \ldots, (X_n, Y_n)$ such that the \emph{excess risk}
\begin{equation*}
	\cE(\wh \theta) := R(\wh \theta) - R(\theta^*)
\end{equation*}
is \emph{minimal}.
Let us first remark that if $(X, Y) \sim P$ with $P \in \cC(P_X, \sigma^2)$ then
\begin{align*}
	\cE(\theta) &= \E [ (Y - X^\top  \theta)^2 - (Y - X^\top \theta^*)^2] \\
	&= \E [ (\theta^* - \theta)^\top X (2 Y - X^\top (\theta + \theta^*))] \\
	&= \E [ (\theta^* - \theta)^\top X (X^\top (\theta + \theta^*) + 2 \eps)] \\
	&= \E [ (\theta^* - \theta)^\top X X^\top (\theta + \theta^*)] = \norm{\theta^* - \theta}_\Sigma^2
\end{align*}
where we used $Y = X^\top \theta^* + \eps$ and $\E(\eps | X) = 0$ and where $\norm{x}_\Sigma^2 = x^\top \Sigma x$ is a norm since we assumed $\Sigma \succ 0$.
Whenever $(X, Y) \sim P$ with $P \in \cC(P_X, \sigma^2)$ we will therefore write
\begin{equation*}
	\cE_P(\theta) = R(\theta) - R(\theta^*) = \norm{\theta^* - \theta}_\Sigma^2
\end{equation*}
and whenever $\wh \theta$ depends on the data $(X_1, Y_1), \ldots, (X_n, Y_n)$ measurable we can consider, since $(X, Y)$ is an independent copy from the data with the same distribution, we can compute
\begin{equation*}
	\E [ \cE_P(\wh \theta) ]
\end{equation*}
where this expectation is with respect to $P^n$, for the randomness coming from the data. We can now consider the \emph{minimax risk} for a set $\cP$ of distributions:
\begin{equation*}
	\inf_{\wh \theta} \sup_{P \in \cP} \E \cE_P(\wh \theta).
\end{equation*}
The infimum is taken over any possible estimator, namely any statistic of the data, while the sup is over all distribution in $\cP$. Hence the name minimax, since we look the worst-case excess risk over the considered set $\cP$, but we consider the best possible estimator (with the inf).
Since $\cG \subset \cC$ we expect the minimax risk of the former to be smaller than the one of the latter.

Some remarks and extra notation is required before we can state the main result of the section.
\begin{itemize}
	\item First, the linear model is \emph{well-specified} here since $Y = X^\top \theta^* + \eps$ with $\E[\eps | X] = 0$, so that there is no approximation term of $\E(Y | X)$ by $X^\top \theta^*$.
This can be done, see for instance ????
\item For the class $\cP = \cC(P_X, \sigma^2)$ we expect a \emph{minimax estimator} $\wh \theta$ that achieves the minimax risk to depend both on $P_X$ and $\sigma^2$. But quite surprisingly, we will see that it won't depend on this knowledge.
\end{itemize}
Let us introduce $\wh \bSigma = \frac 1n \sum_{i=1}^n X_i X_i^\top = \bX^\top \bX / n$ and define the ``whitened'' random vectors $\tilde X_i = \bSigma^{-1/2} X_i$ (so that $\var(\tilde X_i) = \bI_d$) and define
\begin{equation*}
	\tilde \bSigma = \frac 1n \sum_{i=1}^n \tilde X_i \tilde X_i^\top = \bSigma^{-1/2} \wh \bSigma \bSigma^{-1/2}.
\end{equation*}
The following theorem holds.
\begin{theorem}[Mourtada~(2019)]
	Assume that $P_X$ is non-degenerate and $n \geq d$ and $\sigma^2 > 0$. Then
	\begin{equation}
		\inf_{\wh \theta} \sup_{P \in \cC(P_X, \sigma^2)} \E \cE_P(\wh \theta) = \inf_{\wh \theta} \sup_{P \in \cG(P_X, \sigma^2)} \E \cE_P(\wh \theta) = \frac{\sigma^2}{n} \E \tr(\tilde \bSigma ^{-1}).
	\end{equation}
	Furthermore, the infimum in the minimax risk is achieved by the ordinary least squares estimator.
\end{theorem}
This theorem deserves several remarks.
\begin{itemize}
	\item The theorem proves that the least-squares estimator is, in fairly general setting, \emph{minimax optimal}: it cannot be improved by another estimator, uniformly over the class of distributions $\cC(P_X, \sigma^2)$.
	\item The Gaussian noise, namely the class $\cG(P_X, \sigma^2)$ ``saturates'' the minimax risk, and corresponds to the \emph{least favorable} distribution in the minimax sense.
	\item The minimax risk is invariant by a linear transformation of the features vectors: it is unchanged if one replaces $X_i$ by $X_i' = \bA X_i$ for some deterministic invertible matrix $\bA$. Indeed we have in this case $\wh \bSigma' = \frac 1n \sum_{i=1}^n X_i' X_i'^\top = \bA \wh \bSigma \bA^\top$ so that $(\wh \bSigma')^{-1} \Sigma' = (\bA^\top)^{-1} (\wh \bSigma)^{-1} \bA^{-1} \bA \Sigma \bA^\top = (\bA^\top)^{-1} (\wh \bSigma)^{-1} \Sigma \bA^\top$ which proves that $(\wh \bSigma')^{-1} \Sigma'$ and $(\wh \bSigma)^{-1} \Sigma$ are congruent matrices, so that they share the same trace, namely $\tr((\wh \bSigma')^{-1} \Sigma') = \tr((\wh \bSigma)^{-1} \Sigma)$, and the minimax risk is indeed invariant when replacing $X_i$ by $\bA X_i$. This is of course expected, since the supremum is over linear functoin, so invariance with respect to a invertible linear transformation is expected.
\end{itemize}
Since $\bA \mapsto \tr( \bA^{-1})$ is a \emph{convex} function on the set of positive definite matrices, we known from Jensen's inequality that
\begin{equation*}
	\E [\tr( \tilde \bSigma^{-1})] \geq \tr (\E [\tilde \bSigma)] )^{-1}
\end{equation*}
but $\E[ \tilde \bSigma] = \bSigma^{-1/2} \E [\wh \bSigma^{-1}] \bSigma^{-1/2} = \bI_d$ so $\E [\tr( \tilde \bSigma^{-1})] \geq d$ and consequently the minimax risk is larger than $\sigma^2 d / n$.

The proof of the convexity of $\bA \mapsto \tr[ (\bA)^{-1} ]$ can be easily deduced from the following Taylor expansion. Take $\bA \succ 0$ and $\bB$ a symmetrical matrix.
We have
\begin{equation}
	(\bA + t \bB)^{-1} = (\bA (\bI + t \bA^{-1} \bB)^{-1} = \bA^{-1} - t \bA^{-1} \bB \bA^{-1} + t^2 (\bA^{-1} \bB)^2 \bA^{-1} + \ldots
\end{equation}
so that 
\begin{equation*}
	\frac{\partial^2 \tr((\bA + t \bB)^{-1})}{partial t^2} = 2 \tr( (\bA^{-1} \bB)^2 \bA^{-1} ) 
	= 2 \tr( \bA^{-1} \bB \bA^{-1} \bB \bA^{-1} ) = = 2 \tr( \bC \bA^{-1} \bC^\top )
\end{equation*}
where $\bC = \bA^{-1} \bB$. But since $\bA^{-1} \succ 0$ we have $\bC \bA^{-1} \bC^\top \mgeq 0$ which proves that the second derivative is non-negative.

Let us provide now the upper bound part from Theorem~??
We will prove the lower bound later (in Section~???), since it will require some arguments from Bayesian statistics that we did not talked about yet.

\paragraph{Proof of the upper bound frm ???} % (fold)

This is actually mainly a computation with no particular tricks. Recall that $(X, Y)$ is such that $Y = X^\top \theta^* + \eps$ and that $\E(\eps | X) = 0$ and $\E(\eps^2 | X) \leq \sigma^2$.
Consider $\wh \theta$ as the least squares estimator given by
\begin{align*}
	\wh \theta &= (\bX^\top \bX)^{-1} \bX^\top \by = (\bX^\top \bX)^{-1} \bX^\top (\bX \theta^* + \beps) \\
		&= \theta^* + \wh \bSigma^{-1} \frac{1}{n} \sum_{i=1}^n \eps_i X_i
\end{align*}
so that recalling that $\inr{u, v}_{\bSigma} =u^\top \bSigma v$ and
 $\norm{u}_{\bSigma}^2 = \inr{u, u}_{\bSigma}$
\begin{align*}
	\E \cE(\wh \theta) &= \E \Big\| \wh \bSigma^{-1} \frac{1}{n} \sum_{i=1}^n \eps_i X_i \Big \|_{\bSigma}^2
	 = \frac{1}{n^2} \sum_{1 \leq i, i' \leq n} \E \langle \wh \bSigma^{-1} \eps_i X_i, \wh \bSigma^{-1} \eps_{i'} X_{i'} \rangle \\
	 &= \frac{1}{n^2} \sum_{1 \leq i, i' \leq n} \E \Big[ \E [ \eps_i \eps_{i'} | X_1, \ldots, X_n] \langle \wh \bSigma^{-1}  X_i, \wh \bSigma^{-1} X_{i'} \rangle \Big].
\end{align*}
But we have $\E [ \eps_i \eps_{i'} | X_1, \ldots, X_n] = 0$ whenever $i \neq i'$ and $\E [ \eps_i \eps_{i'} | X_1, \ldots, X_n] \leq \sigma^2$ whenever $i=i'$. So, we obtain
\begin{align*}
	\E \cE(\wh \theta) &\leq \frac{\sigma^2}{n^2} \sum_{i=1}^n \E \norm{\wh \bSigma^{-1} X_i}_{\bSigma}^2 
	= \frac{\sigma^2}{n^2} \sum_{i=1}^n \E \big[ (\wh \bSigma^{-1} X_i)^\top \bSigma \wh \bSigma^{-1} X_i \big] \\
	&= \frac{\sigma^2}{n^2} \sum_{i=1}^n \E \big[ \tr (X_i^\top \wh \bSigma^{-1} \bSigma \wh \bSigma^{-1} X_i) \big]
\end{align*}
since $\tr(x) = x$ for $x \in \R$, so that finally, using the cyclic invariance of the trace and linearity, we obtain
\begin{align*}
	\E \cE(\wh \theta) &\leq \frac{\sigma^2}{n^2} \sum_{i=1}^n \E \big[ \tr (\wh \bSigma^{-1} \bSigma \wh \bSigma^{-1} X_i X_i^\top )\big] \\
	&= \frac{\sigma^2}{n} \E \big[ \tr (\wh \bSigma^{-1} \bSigma \wh \bSigma^{-1} \wh \bSigma)\big] \\
	&=\frac{\sigma^2}{n} \E \big[ \tr (\wh \bSigma^{-1} \bSigma)\big] \\
	&=\frac{\sigma^2}{n} \E \big[ \tr ( (\bSigma^{-1/2} \wh \bSigma \bSigma^{-1/2})^{-1}) \big] = \frac{\sigma^2}{n} \E[ \tr (\wt \bSigma^{-1}) ]
\end{align*}
which proves the upper bound.

We can also provide another expression for $\E [\tr (\wt \bSigma^{-1})]$ using the leverage score we talked about in the previous Section~??
Let us recall at this point that since $P_X$ is non-degenerate that if $X_1, \ldots X_{n+1}$ are iid and distributed as $P_X$, we ave that $\sum_{i=1}^{n+1} X_i X_i^\top \succ 0$.
Indeed, the following theorem holds
\begin{theorem}
	\label{thm:minimax-leverage}
	The minimax risk given in Theorem~? above can be written as
	\begin{equation*}
		\frac 1n \E \tr ( (\wt \bSigma)^{-1}) = \E \Big[ \frac{\wh \ell_{n+1}}{1 - \wh \ell_{n+1}} \Big]
	\end{equation*}
	where $\ell_{n+1}$ is the leverage of one data point among $n+1$ given by
	\begin{equation*}
		\wh \ell_{n+1} = X_{n+1}^\top \Big( \sum_{i=1}^{n+1} X_i X_i^\top \Big)^{-1} X_{n+1}
	\end{equation*}
	where $X_1, \ldots, X_n, X_{n+1}$ are iid and with the same distribution as $X$.
\end{theorem}
Let us recall that $\wh \ell_{n+1} = \partial \wh Y_{n+1} / \partial Y_{n+1}$ where $Y_{n+1} = X_{n+1}^\top \wh \theta_{n+1}$ where $\wh \theta_{n+1}$ is the ordinary least squares estimator computed on the $n+1$ samples $(X_1, Y_1), \ldots, (X_{n+1}, Y_{n+1})$. 
This theorem entails that the minimax risk, which measures the complexity of the estimation problem,  is completely determined by the leverage score. 
Even more than that, it is the expected value of a convex functional of it, so that the minimax risk is small with a small leverage, and gets larger with large leverage, which is natural since in such a case, the statistcal problem is harder. 

\paragraph{Proof of TheoremNNN} % (fold)


Let $X_{n+1} \sim P_X$ be independent of $X_1, \ldots, X_n$.
This proof uses a cute trick based on the Sherman-Morrison Lemma \todo{ortho}
\begin{align*}
	\frac 1n \E \tr( \wt \bSigma^{-1}) 
	= \frac 1n \E \tr( (\wh \bSigma)^{-1} \bSigma) 
	= \E \tr( (n \wh \bSigma)^{-1} X_{n+1} X_{n+1}^\top) \\
	= \E \inr{(n \wh \bSigma)^{-1} X_{n+1}, X_{n+1}}.
\end{align*}
We need the following
\begin{lemma}
 For any $\bS \succ 0$ and any $v \in \R^d$ we have
 \begin{equation*}
 	\inr{\bS^{-1} v, v} = \frac{\inr{(\bS + v v^\top)^{-1} v, v}}{1 - \inr{(\bS + v v^\top)^{-1} v, v}}.
 \end{equation*}
\end{lemma}
 BLABLA sur le lemma This lemma gives a nice formula that allows to express a quadratic formula as a function of its rank-1 perturbation BLABLA
\begin{proof}
	We have $\bS + v v^\top \mgeq \bS \succ 0$ so that $\bS + v v^\top$ is invertible so that using the Sherman Morrison formula we have
	\begin{equation*}
		(\bS + v v^\top)^{-1} = \bS^{-1} - \frac{1}{1 + v^\top \bS^{-1} v} \bS^{-1} v v^\top \bS^{-1}
	\end{equation*}
	so that 
	\begin{align*}
		\inr{(\bS + v v^\top)^{-1} v, v} = v^\top \bS^{-1} v - \frac{v^\top \bS^{-1} v v^\top \bS^{-1} v}{1 + v^\top \bS^{-1} v} = \inr{S^{-1} v, v} - \frac{\inr{S^{-1} v, v}^2}{1 + \inr{S^{-1} v, v}} = \frac{\inr{S^{-1} v, v}}{1 + \inr{S^{-1} v, v}}
	\end{align*}
	which concludes the proof of Lemma ???
\end{proof}
\todo{put this lemma somewhere and use it, say that it is useful for many things}

This proves in particular that $\wh \ell_{n+1} \in [0, 1)$ almost surely since $\wh \bSigma \succ 0$ almost surely.
We can know finish the proof of THeorem~? using Lemma ? since now
\begin{align*}
	\frac 1n \E \tr( \wt \bSigma^{-1}) &= \E \inr{(n \wh \bSigma)^{-1} X_{n+1}, X_{n+1}} \\
	&= \E \bigg[ \frac{ \inr{(n \wh \bSigma + X_{n+1} X_{n+1}^\top)^{-1} X_{n+1}, X_{n+1} }}{1 - \inr{(n \wh \bSigma + X_{n+1} X_{n+1}^\top)^{-1} X_{n+1}, X_{n+1} } } \bigg] \\
	&= \E \Big[ \frac{\wh \ell_{n+1}}{1 - \wh \ell_{n+1}} \Big]
\end{align*}
which concludes the proof of THeorem???
% paragraph proof_of_theoremnnn (end)
% paragraph proof_of_the_upper_bound_frm_ (end)

A corollary of Theorem~? is an improved lower bound than the first one we proved $\sigma^2 d / n$.

\begin{corollary}
	Under the same assumptions as THeorem we have that the miimax risk is lower bounded by
	\begin{equation*}
		\sigma^2 \frac{d}{n - d + 1}.
	\end{equation*}
\end{corollary}

\begin{proof}
	Theorem~? and Jensen's inequality since $x \mapsto x / (1-x)$ is convex on $[0, 1]$ gives
	\begin{equation*}
		\E \Big[ \frac{\wh \ell_{n+1}}{1 - \wh \ell_{n+1}} \Big] \geq \frac{\E[\wh \ell_{n+1}]}{1 - \E[\wh \ell_{n+1}}] \Big],
	\end{equation*}
	but a exchangeability arguments gives
	\begin{align*}
		\E [\wh \ell_{n + 1}] &= \E \Big \langle \Big( \sum_{i=1}^{n+1} X_i X_i^\top 
		\Big)^{-1} X_{n+1}, X_{n+1} \Big \rangle \\
		&= \E \Big \langle \Big( \sum_{i=1}^{n+1} X_i X_i^\top \Big)^{-1} X_{i}, X_{i} \Big \rangle
	\end{align*}
	so that
	\begin{align*}
	 	\E [\wh \ell_{n + 1}] &= \frac{1}{n+1} \E \Big \langle \Big( \sum_{i=1}^{n+1} X_i X_i^\top \Big)^{-1} X_{i}, X_{i} \Big \rangle \\
	 	&= \frac{1}{n+1}  \E \tr \bigg( \Big( \sum_{i=1}^{n+1} X_i X_i^\top \Big)^{-1} \sum_{i=1}^{n+1} X_i X_i^\top \bigg) 
	 	&= \frac{d}{n + 1},
	 \end{align*}
	so that
	\begin{equation*}
		\E \Big[ \frac{\wh \ell_{n+1}}{1 - \wh \ell_{n+1}} \Big] \geq \frac{d}{n - d + 1}.
	\end{equation*}
\end{proof}


The $\sigma^2 d / (n - d + 1)$ is very sharp since we know from ??? that whenever $P_X = \nor(0, \bSigma)$ then
\begin{equation*}
	\E \cE_P(\wh \theta) = \sigma^2 \frac{d}{n - d - 1}
\end{equation*}
if $\wh \theta$ is the ordinary least squares estimator. This can be understood also by using Theorem~? (le minimax) together with the use of some knowledge about the Wishart distributions \todo{le faire}.
This means also that the Gaussian ``design'' $\nor(0, \bSigma)$ is almost the most favorable design for linear regression, since for this distribution, the minimax risk is almost minimal (compare denominator ???).
\todo{conjecture that the most favorable is the uniform distribution on the unit sphere}

\paragraph{Upper bound on the minimax risk.} 

We were able to provide easily an explicit lower bound (with respect to $d, n$ and $\sigma^2$) for the minimax risk, it remains to provide a similarly explicit upper bound for this quantity.
\todo{attention y'en a une qui est uniforme $d / n$ et l'autre qui ne l'est pas $d / (n - d + 1$}.

In order to provide such an upper bound, we need some extra technical assumptions on $P_X$.

The first assumption is somehow a quantified version of the non-degenerate assumption about $P_X$. 
Indeed, we assume that there is $\alpha \in (0, 1]$ and $C \geq 1$ such that
\begin{equation}
	\label{eq:quanti-nondegenerate}
	\P[ |\inr{X, \theta} | \geq t \norm{\theta}_{\bSigma} ] \leq (C t)^\alpha
\end{equation}
for any $t > 0$ and non-zero vector $\theta \in \R^d$. Note that this assumption is equivalent to the assumption that $\P[ |\inr{\tilde X, \theta} | \geq t \norm{\theta}_{\bSigma} ] \leq (C t)^\alpha$ for any $\theta \in S^{d-1}$ where we recall that $\wt X = \bSigma^{-1} X$. \todo{sidenote proof}
This assumeption quantifies the assumption $\P(\inr{X, \theta} = 0) = 0$ BLABLA

We need also another assumption that $P_X$ satifies
\begin{equation*}
	\E [\norm{\bSigma^{-1/2} X}^4] \leq \kappa d^2.
\end{equation*}
This is entailed by the $L^4$-$L^2$ condition $\E[\inr{\theta, X}^4]^{1/4} \leq \kappa \E[ \inr{\theta, X}^2]^{1/2}$ for any $\theta \in \R^d$.

Indeed if $\theta = \bSigma^{-1/2} e_j$ then $\inr{\theta, X} = \wt X_j$ so that $\E[\wt X_j^4] \leq \E [\wt X_j^2]^2 = \kappa$ (since $\E[\wt X \wt X^\top] = \bI_d$) so that $\E \norm{\wt X}^4 = \E (\sum_j \wt X_j^2 )^2 = \sum_{1 \leq j, k \leq d} \E[ \wt X_j^2 \wt X_k^2] \leq \sum_{j, k} \sqrt{\E[ \wt X_j^4] \E[] \wt X_k^4]} \leq \kappa d^2$ by assumption. 

\begin{theorem}
	Grant assumptions ?a nd ? and put $C' = 3 C^4 e^{1 + 9 / \alpha}$. If $n \geq 6 \frac d \alpha \vee 12 \log(12 / \alpha) / \alpha$ then
	\begin{equation*}
		\frac 1n \E \tr [ (\wt \bSigma)^{-1}] \leq \frac dn + 8 C' \kappa \Big( \frac dn \Big)^2,
	\end{equation*}
	which entails together with ??? that
	\begin{equation*}
		\sigma^2 \frac dn \leq \inf_{\wh \theta} \sup_{P \in \cC(P_X, \sigma^2)} \E \cE_P(\wh \theta) \leq \sigma^2 \frac dn 
		\Big( 1 + 8 C' \kappa \Big( \frac dn \Big)
	\end{equation*}
\end{theorem}
The proof of this theorem is quite technical and beyond the scope of this book. The interesting reader can read it in ????
The technicity of the proof comes from a sharp control of $\E [\tr (\wt \bSigma)^{-1}]$ or more simply the smallest eigenvalue of $(\wt \bSigma)^{-1}$.

Let us wrap us what are the important things we learned in this Section:
\begin{itemize}
	\item The least-squares procedure is minimax optimal for the well-specified linear regression model
	\item The Gaussian design is almost the most favorable in the minimax sense
	\item The minimax rate is of order $\sigma^2 d / n$
	\item The statistical complexity of the problem of linear regression is, when measured by the minimax rate, fully explained by the distribution a leverage of a sample among $n+1$
\end{itemize}